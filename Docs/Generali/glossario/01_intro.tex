% Insert content here
\section{A}
    \subsection{Approvazione}
    Attività nel quale il documento o parti di files sorgente vengono convalidati dal responsabile di progetto.
    \subsection{Agile}
    Modo di lavoro che si basa sulla suddivisione del progetto in piccoli passi incrementali, in modo 
    tale da avere un feedback il più velocemente possibile e sistemare quindi più rapidamente i problemi.

\section{B}
    \subsection{Branch}
    Feature di github che permette di sviluppare software, nel nostro caso anche documentazione, in modo da creare una suddivisione del lavoro dove ogni modifica non 
    influenzi direttamente il ramo principale.

\section{C}
    \subsection{Commit}
    Sono la parte più importante del sistema di versionamento Git ossia delle istantanee o punti di controllo del progetto in un preciso momento.
    Servono per creare una cronologia permettendo di tornare indietro ad una versione precedente e tracciare ogni azione eseguita da ogni membro del team.

\section{D}
\section{E}
\section{F}
\section{G}
\section{H}
    \subsection{Hotfix}
    E' il numero di versione più a destra e il suo incremento indica: nel caso di un documento che è stata corretta velocemente qualche incoerenza o errore minore, per il codice invece che è stata
    introdotta una correzione ad un semplice bug non invasivo e difficilmente riproducibile.

\section{I}
\section{J}
\section{K}
\section{L}
\section{M}
    \subsection{Major}
    E' il numero di versione più a sinistra e il suo incremento indica: nel caso di un documento che esso è pronto per una delle revisioni ovvero tutte le modifiche precedenti sono state 
    approvate dal responsabile, per il codice invece che è stata introdotta una modifica sostanziale che potenzialmente potrebbe causare problemi di retro-compatibilità.
    \subsection{Minor}
    E' il numero di versione centrale e il suo incremento indica: nel caso di un documento che la sezione aggiunta è stata verificata, per il codice invece che è stata
    introdotta una funzione o feature aggiuntiva che non causa problemi al software oppure una correzione ad un bug complesso è di importanza elevata.

\section{N}
\section{O}
\section{P}
    \subsection{Pages}
    Feature di github che permette tramite un sito web di presentare una repository in un modo più accessibile e ordinato.
    \subsection{Pull request}
    Feature di github che permette di chiedere ad un verificatore o al responsabile se il contenuto di un branch può
    essere unito alla parte principale del progetto, già verificata e pronta per essere mandata in produzione.

\section{Q}
\section{R}
\section{S}
    \subsection{Sprint}
    Periodo di tempo corrispondente 2/3 settimane nel quale il gruppo di lavoro si assegna degli 
    obiettivi, suddivide il lavoro, lo svolge e alla fine verifica i risultati.
    \subsection{Scrum}
    E' uno dei modi di lavoro agile più famosi e utilizzati, il tempo di sviluppo si suddivide in sprint della durata di 2/3 settimane,
    prevede una retrospective dove il team analizza ciò che ha fatto nello sprint precedente e quali sono stati i problemi in modo da migliorarsi
    e una riunione giornaliera per discutere quali sono le attività da svolgere in data odierna.

\section{T}
\section{U}
\section{V}
    \subsection{Verifica}
    Insieme di attività chiamate in causa molte volte dal processo di sviluppo. Ha lo scopo di assicurare
    che un prodotto, sistema o servizio soddisfi i requisiti specificati e ne garantisce la qualità e l’affidabilità
    prevenendo eventuali problemi che possono verificarsi durante il suo utilizzo.

\section{W}
    \subsection{Way of working}
    Termine che riassume tutte le regole di lavoro e organizzazione del team.

\section{X}
\section{Y}
\section{Z}
