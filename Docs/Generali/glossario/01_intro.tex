% Insert content here
\section{A}
    \subsection{Agile}
    Approccio alla gestione dei progetti che prevede la suddivisione in fasi e sottolinea 
    l'importanza della collaborazione e del miglioramento continuo tramite feedback. 
    Il team segue un ciclo di pianificazione, esecuzione e valutazione.
    \subsection{API}
    Abbreviazione per \textit{Application Programming Interface}, indica un insieme di 
    procedure che consentono la comunicazione tra diversi computer, tra diversi 
    software o tra diversi componenti di software.
    \subsection{Approvazione}
    Attività nella quale un documento viene convalidato dal responsabile di progetto.
    \subsection{Assignee}
    Letteralmente "Assegnatario", ovvero membro del team a cui è assegnato lo
    svolgimento di una determinata attività.
\section{B}
    \subsection{Backlog}
    Lista prioritaria e ordinata di attività, requisiti o funzionalità che devono 
    essere completati per sviluppare un prodotto o un progetto.
    Vedi anche Product backlog e Sprint backlog.
    \subsection{Branch}
    In Git, ramo del progetto che permette di sviluppare software, documentazione inclusa, su una linea separata 
    senza interferire con il ramo principale o altri rami del repository.

\section{C}
    \subsection{Capitolato}
    Documento tecnico a cui si fa riferimento per definire le specifiche tecniche 
    del prodotto che verrà realizzato. Può contenere indicazioni sulle metodologie e
    tecnologie da adottare per lo sviluppo del prodotto.
    \subsection{Commit}
    In Git, istantanea di uno specifico stato del tuo progetto in un determinato momento.
    Ogni commit ha un codice univoco che lo identifica e include le modifiche apportate ai file,
    un riferimento allo stato del repository prima del commit e un messaggio descrittivo fornito dallo sviluppatore.
    \subsection{Committente}
    Chi ordina ad altri l'esecuzione di un lavoro. Nel caso del progetto didattico
    è il docente.
    \subsection{Consuntivo}
    Rendiconto dei risultati effettivi di un dato periodo di attività, 
    in termini di ore e costi, messi a confronto con gli obiettivi pianificati.
\section{D}
    \subsection{Daily scrum}
    Incontro giornaliero a cui partecipano tutti i membri del team, in cui ciascuno
    risponde alle seguenti domande:
    \begin{itemize}
        \item "Cosa hai fatto ieri?"
        \item "Cosa farai oggi?"
        \item "C'è qualcosa che ti impedisce di farlo?"
    \end{itemize}
    \subsection{Database}
    Archivio di dati strutturato in modo da consentire la gestione e 
    l'organizzazione dei dati.
    \subsection{Diagramma di Gantt}
    Strumento di supporto alla gestione dei progetti, costruito partendo da 
    un asse orizzontale, che rappresenta l'arco temporale totale del progetto, 
    suddiviso in fasi incrementali (ad esempio giorni o settimane), e da un asse 
    verticale, che rappresenta le attività che costituiscono il progetto.
    \subsection{Discord}
    Piattaforma di VoIP(Voice over Internet Protocol), messaggistica istantanea 
    e distribuzione digitale, progettata per la comunicazione tra gruppi di persone.
\section{E}
    \subsection{Efficacia}
    Capacità di raggiungere gli obiettivi attesi.
    \subsection{Efficienza}
    Capacità di utilizzare la minima quantità di risorse necessarie al raggiungimento
    di un obiettivo.
\section{F}
    \subsection{Feature}
    Vedi Funzionalità.
    \subsection{Feedback}
    Informazione di ritorno, opinione o valutazione che viene fornita come reazione a 
    un'attività o un prodotto, al fine di migliorare o correggere i risultati futuri.
    \subsection{Fornitore}
    Chi si impegna a svolgere un lavoro o realizzare un prodotto. Nel caso del progetto didattico
    è il gruppo di studenti.
    \subsection{Framework}
    \subsection{Funzionalità}
    Specifica caratteristica o capacità di un prodotto, software, applicazione o sistema, 
    che ha lo scopo di soddisfare un'esigenza o risolvere un problema per l'utente.
\section{G}
    \subsection{Git}
    Sistema software distribuito che tiene traccia delle versioni dei file. 
    Viene utilizzato per controllare il codice sorgente da parte di programmatori 
    che stanno sviluppando software in modo collaborativo.
    \subsection{GitHub}
    Piattaforma per sviluppatori che permette di creare, memorizzare, gestire 
    e condividere il loro codice. È utilizzato principalmente per ospitare 
    progetti di sviluppo software open source.
    \subsection{GitHub Pages}
    Servizio offerto da GitHub che permette di pubblicare siti web statici direttamente da un repository GitHub,
    usato per ospitare la documentazione prodotta.
    \subsection{Google Chat}
    Strumento di comunicazione con funzionalità di messaggistica
    istantanea e condivisione di file, integrato con altri servizi Google.
    \subsection{Google Meet}    
    Servizio web per videoconferenze e riunioni, con funzionalità di messaggistica,
    condivisione schermo e registrazione.
\section{H}
    \subsection{Hotfix}
    Correzione urgente applicata a un sistema software per risolvere un problema critico, 
    che richiede un intervento immediato. A differenza delle normali modifiche o aggiornamenti pianificati, 
    un hotfix viene sviluppato e distribuito rapidamente per minimizzare l'impatto del problema.\\
    In caso di hotfix di un documento, viene aumentato di uno il numero di versione più a destra.

\section{I}
    \subsection{Issue}
    Strumento utilizzato per tracciare, discutere e risolvere problemi, richieste di funzionalità, 
    bug o idee relative a un progetto, facilitando la collaborazione tra i membri di un team.
\section{J}
\section{K}
\section{L}
    \subsection{LaTeX}
    Linguaggio di markup utilizzato per la preparazione di documenti di alta qualità tipografica.
\section{M}
    \subsection{Main}
    Branch principale di sviluppo di un progetto software.
    \subsection{Major}
    Numero di versione più a sinistra, il cui incremento indica: nel caso di un documento, che esso è pronto per una delle revisioni, ovvero tutte le modifiche precedenti sono state 
    approvate dal responsabile; per il codice, invece, che è stata introdotta una modifica sostanziale che potenzialmente potrebbe causare problemi di retrocompatibilità.
    \subsection{Merge}
    In Git, processo di combinazione di modifiche provenienti da un branch (tipicamente 
    un ramo secondario) in un altro (tipicamente il main).
    \subsection{Milestone}
    Data che fissa un punto di avanzamento atteso nello svolgimento del progetto.
    \subsection{Minor}
    Numero di versione centrale, il cui incremento indica: nel caso di un documento, che la sezione aggiunta è stata verificata; per il codice, invece, che è stata
    introdotta una funzione o feature aggiuntiva che non causa problemi al software preesistente, oppure una correzione ad un bug complesso di importanza elevata.

\section{N}
\section{O}
    \subsection{Overleaf}
\section{P}
    \subsection{Pages}
    Vedi GitHub Pages.
    \subsection{PoC}
    Vedi Proof of Concept.
    \subsection{Preventivo}
    \subsection{Proof of Concept}
    \subsection{Proponente}
    \subsection{Pull request}
    Funzionalità comune nelle piattaforme di sviluppo collaborativo, come GitHub, utilizzata per 
    richiedere l'integrazione del lavoro di uno sviluppatore (di solito su un branch separato) 
    nel ramo principale (main o master) o in un altro ramo del repository.
    \subsection{Push}

\section{Q}
\section{R}
    \subsection{Repository}
    \subsection{Retrospettiva}
    Vedi Sprint retrospective.
\section{S}
    \subsection{Sprint}
    Breve periodo di tempo in cui un team Scrum collabora per completare 
    una determinata quantità di lavoro. Uno sprint comprende:  
    sprint planning, daily scrum, lavoro di sviluppo, 
    sprint review e sprint retrospective. Il team \textit{Six Bix Busters} ha 
    deciso che ogni sprint durerà due settimane.   
    \subsection{Sprint planning}
    \subsection{Sprint retrospective}
    \subsection{Sprint review} 
    \subsection{Scrum}
    Framework agile per la gestione del ciclo di sviluppo del software, 
    iterativo ed incrementale, concepito per gestire progetti e prodotti software 
    o applicazioni di sviluppo.

\section{T}
    \subsection{Task}
    \subsection{Team}
    \subsection{Telegram}
\section{U}
\section{V}
    \subsection{Verifica}
    Processo che determina se i prodotti software di un'attività soddisfano
    i requisiti o le condizioni imposte nelle attività precedenti.
    \subsection{Versionamento}

\section{W}
    \subsection{Way of working}
    Il modo in cui il gruppo decide di lavorare, ovvero l'insieme di regole 
    che il gruppo si dà al fine di organizzare al meglio le attività di progetto.
    \subsection{Workflow}
\section{X}
\section{Y}
\section{Z}
